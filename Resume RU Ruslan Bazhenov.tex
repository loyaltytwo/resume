%%%%%%%%%%%%%%%%%%%%%%%%%%%%%%%%%%%%%%%%%
% Medium Length Professional CV
% LaTeX Template
% Version 2.0 (8/5/13)
%
% This template has been downloaded from:
% http://www.LaTeXTemplates.com
%
% Original author:
% Trey Hunner (http://www.treyhunner.com/)
%
% Important note:
% This template requires the resume.cls file to be in the same directory as the
% .tex file. The resume.cls file provides the resume style used for structuring the
% document.
%
%%%%%%%%%%%%%%%%%%%%%%%%%%%%%%%%%%%%%%%%%

%----------------------------------------------------------------------------------------
%	PACKAGES AND OTHER DOCUMENT CONFIGURATIONS
%----------------------------------------------------------------------------------------

\documentclass{resume} % Use the custom resume.cls style

\usepackage[left=0.75in,top=0.6in,right=0.75in,bottom=0.6in]{geometry} % Document margins

\name{Руслан Баженов} % Your name
\address{Алматы, Казахстан} % Your address
% \address{123 Pleasant Lane \\ City, State 12345} % Your secondary addess (optional)
\address{+7(707)516-1199 \\ ruslan.bazhenov@nu.edu.kz} % Your phone number and email

\begin{document}

%----------------------------------------------------------------------------------------
%	EDUCATION SECTION
%----------------------------------------------------------------------------------------

\begin{rSection}{Образование}

{\bf Назарбаев Университет, Нур-Султан} \hfill {\em Июнь 2021} \\ 
Бакалавр Электрической и Электронной Инженерии \\
\smallskip \\
Член правления студенческой организации IEEE  \\
Средний балл: 3.46 / 4.00

\end{rSection}

%----------------------------------------------------------------------------------------
%	WORK EXPERIENCE SECTION
%----------------------------------------------------------------------------------------

\begin{rSection}{Опыт Работы}

\begin{rSubsection}{АО "First Heartland Jusan Bank"\ }{Апрель - Июль 2021}{Инженер Машинного Обучения}{Алматы}
\item Младший специалист по науке о данных в департаменте анализа данных и машинного обучения. Разработка нейронных сетей и применение науки о данных для улучшения качества услуг оказываемых банком.
\end{rSubsection}

%------------------------------------------------

\begin{rSubsection}{ОЮЛ "ЦАРКА"\ }{Май - Август 2019}{Стажёр}{Нур-Султан}
\item Центр Анализа и Расследования Кибер-Атак
\item Разработка проектов на базе микроконтроллеров STM32, Arduino, Raspberry. Сбор и агрегация данных при помощи Splunk.
\end{rSubsection}

%------------------------------------------------

\begin{rSubsection}{STEM Академия}{Октябрь 2018 - Январь 2019}{Преподаватель робототехники и программирования}{Нур-Султан}
\item Преподавание робототехники и программирования на языке Python детям 8-12 лет. Участвовал в организации региональной олимпиады школьников.
\end{rSubsection}

%------------------------------------------------

\begin{rSubsection}{АО "НИПИнефтегаз"\ }{Июнь - Август 2018}{C\# Разработчик}{Актау}
\item Разработка ПО для внутренних нужд IT департамента на языке C\#, .NET Framework. Работа с Active Directory. 
\end{rSubsection}

\end{rSection}

%----------------------------------------------------------------------------------------
%	CERTIFICATES SECTION
%----------------------------------------------------------------------------------------

\begin{rSection}{Сертификаты и Награждения}


    {\bf Лидерство в Области Науки о Данных} \hfill {\em Май 2021} \\ 
    Стэнфордский Университет, Высшая Школа Бизнеса
    
    {\bf Участие в Сессиях Коллегиального Обучения} \hfill {\em Ноябрь 2017} \\
    Назарбаев Университет, Школа Инженерии

    {\bf Участие в Организации "ABC Hack"\ хакатона} \hfill {\em Сентябрь 2017} \\
    Назарбаев Университет, Система Исследований и Инноваций

    {\bf Бронзовая медаль на Республиканской Олимпиаде по Информатике} \hfill {\em Март 2017} \\
    Министерство Образования и Науки Республики Казахстан

\end{rSection}

%----------------------------------------------------------------------------------------
%	TECHNICAL STRENGTHS SECTION
%----------------------------------------------------------------------------------------

\begin{rSection}{Навыки}

\begin{tabular}{ @{} >{\bfseries}l @{\hspace{6ex}} l }
Языки программирования & Python, C/C++, C\#, .NET, Java \\
Протоколы \& APIs & XML, JSON, REST \\
Базы данных & MySQL, PostgreSQL, SQLite, MongoDB \\
ML инструменты & TensorFlow, PyTorch, NumPy, Pandas, OpenCV \\
Прочие инструменты & Linux, Git, GitLab, MLflow, JIRA, Slack, Docker
\end{tabular}

\end{rSection}

%----------------------------------------------------------------------------------------
%	EXAMPLE SECTION
%----------------------------------------------------------------------------------------

%\begin{rSection}{Section Name}

%Section content\ldots

%\end{rSection}

%----------------------------------------------------------------------------------------

\end{document}
